\documentclass[a4paper]{article}
\usepackage[utf8]{inputenc}
\usepackage[T1]{fontenc}
\usepackage{lmodern}
\usepackage[francais]{babel}
\usepackage{fullpage}
\usepackage[hidelinks]{hyperref}
\usepackage{graphicx}
\usepackage{amsmath,amsfonts,amssymb}
\usepackage{tikz}
\usepackage{eurosym}

\begin{document}
  \title{Évaluation de Mathématiques}
  % \author{...}
  \date{
    Nom~: .......................................\\
    \vspace{0.2cm}
    Date~: .......................................}
  \maketitle

  \section*{Exercice 1}
  \paragraph{3 points}
  On donne l'expression suivante:
  \[
    A = 9 x^2 - 25 - 2(3x - 5)
  \]
  \paragraph{1)} Développer~$A$.
  \paragraph{2)} Factoriser~$9 x^2 - 25$.
  \paragraph{3)} Factoriser~$A$.

  \subsection*{Question bonus}
  \paragraph{2 points}
  Soit:
  \[
    f(x) = 3x - 5
  \]
  Donner un antécédent par la fonction~$f$ du nombre~$-8$.

  \section*{Exercice 2}
  \paragraph{4 points}
  Déterminer le facteur commun et factoriser chaque expression.
  \[
    \begin{array}{rcl}
      A &=& (2x + 1)^2 + (3x - 8) (2x + 1)\\\\
      B &=& (3x - 1) - (3x - 1)(5x + 8)\\\\
      C &=& (x + 5)(8 - 3x) + 2(x + 5)\\\\
      D &=& (x - 5)(2x + 1) - 2(5 - x)
    \end{array}
  \]

  \section*{Exercice 3}
  \paragraph{3 points}
  Factoriser chaque expression en utilisant les identités remarquables.
  \[
    \begin{array}{rcl}
      A &=& x^2 - 16\\\\
      B &=& x^2 - 2x + 1\\\\
      C &=& x^2 + 6x + 9
    \end{array}
  \]

  \section*{Exercice 4}
  \paragraph{3 points}
  Développer chaque expression en utilisant les identités remarquables.
  \[
    \begin{array}{rcl}
      A &=& (x + 3)^2)\\\\
      B &=& (x - 4)(x + 4)\\\\
      C &=& (x - 1)^2
    \end{array}
  \]

  \section*{Exercice 5}
  \paragraph{3 points}
  Résoudre les équations suivantes:
  \paragraph{1)} $3(x - 1) + 2(x + 3) = 0$
  \paragraph{2)} $-5(x + 2) + 3(x - 3) = 7$
  \paragraph{3)} $5(2x - 1) = 3(2x + 5)$
  \paragraph{4)} $(3x + 3)(2x - 4) = 0$ \quad (uniquement pour la $3^\text{ème}$ E)
\end{document}
